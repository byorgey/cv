\documentclass{article}

\pagestyle{empty}

\begin{document}

\noindent Brent Yorgey \\
Statement of teaching philosophy \\
\today
\bigskip

My primary job as a teacher is not to impart knowledge, but to create
\emph{welcoming, encouraging communities of learning} in which students can
have \emph{transformative encounters with beautiful ideas}.  Learning
does not take place in a vacuum, but is fundamentally relational.

Welcome: the sad fact in computer science is that many students never
even make it in the door.  Of course this applies particularly in the
case of women and minorities, XXX
I strive to make my courses into welcoming communities---where
students feel they can belong, and indeed that there is something to
belong to.  XXX Mention CIS 399. Students help each other learn.

Failure is unavoidable in any human endeavor, and especially so in an
environment emphasizing creativity and exploration.  XXX? I also
recognize my own constant failings as a teacher, as I explore subjects
along with my students and discover (sometimes successfully and
sometimes less so) how best to guide them.  This recognition informs
how I strive to interact with students experiencing failure of one
sort or another, from a missed question to a missed semester: fairly,
not shying away from the failure or its consequences, but with
compassion, grace, and second chances.

This philosophy comes across in my classroom in several ways. First, I
try to illustrate topics in fun and creative ways: with an appropriate
picture or diagram, with props, with audience participation.  I've
taught middle school students about finite automata using magic
markers and giant sheets of paper on the floor for them to draw and
walk around on.  I've taught recursion using a bowl full of marbles
and a ball of twine.  I've had a confederate ``interrupt'' class with
an apparently inane request which later turned out to illustrate a key
concept.  I use slides when necessary but keep them as simple as
possible, with very little text and lots of illustrations.  In short,
I try to stimulate many different modes of learning, and help students
develop intuition for big ideas, rather than getting lost in a maze of
facts.

Second, I emphasize self-discovery. Students remember ideas they
discover for themselves, because they understand the motivation and
how the idea fits into a larger context. I didn't tell the middle
school students that they were learning about finite automata---I just
let them play! The sorts of content-driven, lecture-based courses I
teach can make this difficult, but I try my best to engage students'
faculties of discovery by assigning open-ended projects, encouraging
questions, and interspersing lectures with small-group or individual
problem solving sessions.  I often introduce new topics in the form of
a challenging problem to be solved, then sit back and watch the
students hash out a solution among themselves.  Ten minutes of false
starts, arguments, insights, and eventual triumph are worth more than
ten hours of lecturing!

%The students who succeed are the ones who take on something
%unfamiliar to them, explore it

Above all, I try to convey an appreciation for beautiful ideas,
not to receptacles of knowledge but to human beings. A student will
almost certainly forget all the details of a proof or definition or
technique.  But if they remember the beautiful nugget of insight it
represents, and where it fits within the larger context of the
subject---then I will have truly taught them something.

% In college I took a number theory course which strongly influenced my
% ideas about teaching.  
% 
% Although I have never done so, I would also love to try
% teaching a course using a modification of the Moore method, pioneered
% by Robert Lee Moore at the beginning of the 20th century while a
% mathematics professor at the University of Pennsylvania. He explicitly
% taught his students as little as possible, only stating definitions
% and theorems and having the students work out the proofs for
% themselves. I once took an undergraduate number theory course taught
% in a similar style, and I still remember more from that course than
% from four or five other average courses put together!

My extraordinary commitment to teaching is evidenced by the way I have
consistently gone beyond expectations and requirements, and the awards
I have won.  Computer science PhD students at Penn are only required
to serve as a teaching assistant for two semesters; during my time at
Penn I served as a teaching assistant for three semesters, choosing
courses where I would have real teaching duties rather than simply
grading.  I was awarded a Teaching Practicum Award, given every year
to several PhD students in the CS department in recognition of their
outstanding effort and enthusiasm as a TA.  However, I didn't stop
there---I went on to design and teach two other undergraduate courses
myself. The first, a half-credit introduction to the Haskell
programming language, I taught three times, XXX popular course.  I
also created an experimental full-credit course entitled ``The Art of
Recursion'', exploring the theory and practice of recursion from a
number of viewpoints.  My efforts were recognized with the Penn Prize
for Excellence in Teaching by Graduate Students, a prestigious,
student-nominated award given yearly to ten graduate students selected
from across all of Penn's graduate programs.

I take every opportunity to reflect upon and improve my
teaching. During my time at Penn I obtained a teaching certificate
through Penn's Center for Teaching and Learning, which involved
participation in a number of workshops, a formal teaching observation,
and a capstone seminar.  I am also an active participant in an
informal seminar for graduate students interested in CS education,
where we discuss various issues in education and give each other
feedback and encouragement in our teaching.



\end{document}
