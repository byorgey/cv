\documentclass{article}

\pagestyle{empty}

\begin{document}

\noindent Brent Yorgey \\
Statement of teaching philosophy \\
\today
\bigskip

As a teacher, my primary job is not to impart knowledge, but to create
\emph{welcoming communities of learning} in which students can have
\emph{transformative encounters with beautiful ideas}.  I am
extraordinarily committed to teaching; teaching and research both
bring me great joy as inseparable halves of a unified academic
pursuit.

A \emph{welcoming community of learning} is a social space where
students feel acceptance and true belonging in a group of people who
are passionate about learning rather than promting themselves.  The
sad fact in computer science is that many students never feel this
sense of belonging; of course this applies particularly in the case of
women and minorities. This sense of belonging---or not belonging---can
come from many places, but teachers play a critical role.  I strive to
make my courses into such welcoming communities---places where every
student feels they can belong, and indeed that there is something to
belong to.  In a course I designed and taught at Penn, ``The Art of
Recursion'', I tried to foster this kind of community in several ways.
First, I used the unusual format of alternating lectures and
student-led discussions. Every week we devoted an entire class period
to having randomly chosen students present their solutions to homework
problems and then lead the class in discussing them.  I tried to stay
out of the way as much as possible and let the students hash out
solutions on their own.  On its own, one might think this format would
create anxiety and feelings of being judged---precisely the opposite
of what I wanted.  The difference is that we spent time at the
beginning of the semester explicitly discussing the sort of community
we wanted to create, and how to go about it: to value one another as
human beings; to make a clear separation between that value and the
mistakes and failures inherent in our pursuit of truth; and to
contribute in ways designed to promote others instead of oneself. I
also tried hard to model all of this for them. In the end, it resulted
in an amazing amount of respectful interaction, discussion, and
collaboration. From my perspective, it seemed that the
students---including several women---felt a strong sense of shared
accomplishment and belonging.

% Another way I create welcoming communities is by the way that I
% acknowledge and deal with failure.  Failure is unavoidable in any
% human endeavor, and especially so in an environment emphasizing
% creativity and exploration.  XXX? I also recognize my own constant
% failings as a teacher, as I explore subjects along with my students
% and discover (sometimes successfully and sometimes less so) how best
% to guide them.  This recognition informs how I strive to interact with
% students experiencing failure of one sort or another, from a missed
% question to a missed semester: fairly, not shying away from the
% failure or its consequences, but with compassion, grace, and second
% chances.

Once a community of learning is created, it would be a waste to simply
impart information; indeed, students to whom information is merely
imparted will never feel a part of a community in the first
place. Marvelling at beautiful ideas is one my biggest personal
motivations in both research and teaching, and my goal is to afford
students the opportunity to have \emph{transformative encounters with
  beautiful ideas}.  This comes across in my classroom in a number of
ways.  I try to illustrate topics in fun and creative ways: with an
appropriate picture or diagram, with props, with audience
participation.  I've taught middle school students about finite
automata using magic markers and giant sheets of paper on the floor.
I've taught recursion using a bowl full of marbles and a ball of
twine.  I try to stimulate many different modes of learning, and help
students develop intuition for big ideas, rather than getting lost in
a maze of facts.  Above all, I find ways to get excited about the
material I am teaching and let my enthusiasm and wonder come across in
my teaching.

Wanting students to encounter beautiful ideas means that I also
emphasize self-discovery. I am a big proponent of ``inquiry-based
learning'' approaches, where students are the primary drivers of the
learning process. Students remember ideas they discover for
themselves, because they understand the motivation and how the idea
fits into a larger context. I didn't tell the middle school students
that they were learning about finite automata---I just let them play!
The sorts of content-driven, lecture-based courses I teach can make
this difficult, but I try my best to engage students' faculties of
discovery by assigning open-ended projects, encouraging questions, and
interspersing lectures with small-group or individual problem solving
sessions.

%  I often introduce new topics in the form of a challenging
% problem to be solved, then sit back and watch the students hash out a
% solution among themselves.  Ten minutes of false starts, arguments,
% insights, and eventual triumph are worth more than ten hours of
% lecturing!

% Above all, I try to convey an appreciation for beautiful ideas,
% not to receptacles of knowledge but to human beings. A student will
% almost certainly forget all the details of a proof or definition or
% technique.  But if they remember the beautiful nugget of insight it
% represents, and where it fits within the larger context of the
% subject---then I will have truly taught them something.

Finlly, I am extraordinarily committed to teaching and to continually
improving as a teacher.  Computer science PhD students at Penn are
required only to serve as a teaching assistant for two semesters; I
not only served as a teaching assistant for three semesters, but also
designed and taught two of my own courses from scratch, and completed
a teaching certificate through Penn's Center for Teaching and
Learning.  As a TA, I chose courses where I would have real teaching
duties rather than simply grading, contributing particularly to the
design of Benjamin Pierce's experimental ``Software Foundations''
course. I was awarded a Teaching Practicum Award, given every year to
several PhD students in the CS department in recognition of their
outstanding effort and enthusiasm as a TA.  I also went on to design
and teach two undergraduate courses myself. The first, a half-credit
introduction to the Haskell programming language, I taught three
times.  By the third iteration it had become quite popular, requiring
me to turn some students away.  I also created an experimental
full-credit course entitled ``The Art of Recursion'', exploring the
theory and practice of recursion from a number of viewpoints.  My
efforts were recognized with the Penn Prize for Excellence in Teaching
by Graduate Students, a prestigious, student-nominated award given
yearly to ten graduate students selected from across all of Penn's
graduate programs.  Finally, concurrently with all of this I completed
a teaching certificate through Penn's Center for Teaching and
Learning, which required participation in a number of workshops, a
formal teaching observation, and a capstone seminar.  I have also been
an active participant in an informal seminar for graduate students
interested in CS education, where we discuss various issues in
education and give each other feedback and encouragement in our
teaching.



\end{document}
