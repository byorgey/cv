\documentclass[12pt]{article}

\usepackage[margin=0.9in,noheadfoot]{geometry}
\usepackage{natbib}

\pagestyle{empty}

\begin{document}

\noindent Brent Yorgey \\
Statement of teaching philosophy \\
\today
\bigskip

As a teacher, my primary job is not to impart facts, but to create
\emph{welcoming communities of learning} in which students can have
\emph{transformative encounters with beautiful ideas}.  I am
extraordinarily committed to teaching; teaching and research both
bring me great joy as inseparable halves of a unified academic
pursuit.

A \emph{welcoming community of learning} is a social space where
students feel acceptance and true belonging in a group of people
passionate about learning rather than promoting themselves.  The sad
fact in computer science is that many students---particularly, it
seems, women and minorities---never feel this sense of
belonging. Teachers have a critical role to play, in shaping the
culture and social spaces in which students learn.  I strive to make
my courses into welcoming communities, places where every student
feels they can and do belong, and indeed that there is something to
belong to.  In a course I designed and taught at the University of
Pennsylvania, ``The Art of Recursion'', I tried to foster this kind of
community in several ways.  First, I used the unusual format of
alternating lectures and student-led discussions. Every week we
devoted an entire class period to having randomly chosen students
present their solutions to homework problems and then lead the class
in discussing them.  I tried to stay out of the way as much as
possible and let the students hash out solutions on their own.  Of
course, with no other structure, this format has the potential to
create anxiety and feelings of being judged, just the opposite of a
welcoming community.  The difference is that we spent time at the
beginning of the semester explicitly discussing the sort of community
we wanted to create, and how to go about it: to value one another as
human beings; to make a clear separation between that value and the
mistakes and failures inherent in our pursuit of truth; and to
contribute in ways designed to promote others instead of oneself. I
also tried hard to model all of this for them. In the end, it resulted
in an amazing amount of respectful interaction, discussion, and
collaboration. From my perspective, it seemed that the
students---including several women---felt a strong sense of shared
accomplishment and belonging.

At Williams, where I have been co-teaching an introductory programming
course, I have seen all too clearly the ways in which some students
feel that they are not good enough, that they do not belong, that they
will be found out as an imposter.  I try to make clear in my
interactions with students that they can and do belong and that I
value them and their uniquenesses more highly than I value their
performance in class.  I have also begun thinking of creative ways to
foster a positive sense of belonging in the second-semester CS course
which I will teach in the spring, including doing some variant on the
celebrated “values exercise” \citep{miyake2010reducing} on the first
day of class.

% Another way I create welcoming communities is by the way that I
% acknowledge and deal with failure.  Failure is unavoidable in any
% human endeavor, and especially so in an environment emphasizing
% creativity and exploration.  XXX? I also recognize my own constant
% failings as a teacher, as I explore subjects along with my students
% and discover (sometimes successfully and sometimes less so) how best
% to guide them.  This recognition informs how I strive to interact with
% students experiencing failure of one sort or another, from a missed
% question to a missed semester: fairly, not shying away from the
% failure or its consequences, but with compassion, grace, and second
% chances.

Once an inclusive community of learning is created, it would be a
waste to simply impart facts; indeed, students to whom facts are
merely imparted will never feel a part of a community in the first
place. Marvelling at beautiful ideas is one my biggest personal
motivations in both research and teaching, and my goal is to afford
students the opportunity to have \emph{transformative encounters with
  beautiful ideas}.  I want students to come away not just as better
critical thinkers and problem solvers, but as better problem
\emph{posers}, and as better connoisseurs of problems and solutions.
The ability to pose good questions, and to appreciate good solutions,
gets at the heart of the creative and scientific processes.  I use a
number of techniques to achieve this.  First, I try to illustrate
topics in fun and creative ways, not necessarily to entertain
students, but to help them get beyond their preconceived ideas and
encounter computer science and mathematics in new ways.  I've taught
middle school students about finite automata using magic markers and
giant sheets of paper on the floor; I've taught recursion using a bowl
full of marbles and a ball of twine; I've had a confederate
``interrupt'' class with an inane request that turned out to
illustrate a key concept.  I also typically spend a good deal of class
time talking about intuition and the big picture, trying to help
students see where the current topic sits within a broader context and
how it connects to others.  Finally, I am a big proponent of
``inquiry-based learning'' approaches, where students are the primary
drivers of the learning process. Students remember ideas they discover
for themselves, because they understand the motivation and how the
idea fits into a larger context.

%  I often introduce new topics in the form of a challenging
% problem to be solved, then sit back and watch the students hash out a
% solution among themselves.  Ten minutes of false starts, arguments,
% insights, and eventual triumph are worth more than ten hours of
% lecturing!

% Above all, I try to convey an appreciation for beautiful ideas,
% not to receptacles of knowledge but to human beings. A student will
% almost certainly forget all the details of a proof or definition or
% technique.  But if they remember the beautiful nugget of insight it
% represents, and where it fits within the larger context of the
% subject---then I will have truly taught them something.

Finally, I am extraordinarily committed to teaching and to continually
improving as a teacher.  During my time as a graduate student at the
University of Pennsylvania, I not only served as a teaching assistant
for three semesters (only two are required), but also designed and
taught two of my own courses from scratch, and completed a teaching
certificate through Penn's Center for Teaching and Learning.  As a TA,
I chose courses where I would have real teaching duties rather than
simply grading, contributing particularly to the design of Benjamin
Pierce's experimental ``Software Foundations'' course. I was awarded a
Teaching Practicum Award, given every year to several PhD students in
the CS department in recognition of their outstanding effort and
enthusiasm as a TA.  I went on to teach my own undergraduate course, a
half-credit, semester-long introduction to the Haskell programming
language, offering it a total of three times. The materials I
developed for that course are now some of the materials most highly
recommended by the Haskell community to people wishing to learn
Haskell.  As described previously, I also created an experimental
full-credit course entitled ``The Art of Recursion'', exploring the
theory and practice of recursion from a number of viewpoints.  My
efforts were recognized with the Penn Prize for Excellence in Teaching
by Graduate Students, a prestigious, student-nominated award given
yearly to ten graduate students selected from across all of Penn's
graduate programs.  Finally, I concurrently completed a teaching
certificate through Penn's Center for Teaching and Learning, which
required participation in a number of workshops, a formal teaching
observation, and a capstone seminar.  I was also an active participant
in an informal seminar for graduate students interested in CS
education, where we discussed selected issues in education and gave
each other feedback and encouragement in our teaching.

Courses I am willing and able to teach include introductory courses,
data structures, algorithms, programming languages, theory of
computation, compilers, discrete mathematics, and linear algebra. I
would also be excited to offer advanced courses in functional
programming or category theory.
\nopagebreak
\bibliographystyle{plain}
\bibliography{teaching-statement}

\end{document}
