\documentclass{article}

\pagestyle{empty}

\begin{document}

\noindent Brent Yorgey \\
Statement of teaching philosophy \\
\today
\bigskip

My teaching philosophy revolves around a single premise: people
don't remember facts; they remember beautiful ideas that help them
make sense of the world.

This philosophy comes across in my classroom in several ways. First, I
try to illustrate topics in fun and creative ways: with an appropriate
picture or diagram, with props, with audience participation.  I've
taught middle school students about finite automata using magic
markers and giant sheets of paper on the floor for them to draw and
walk around on.  I've taught recursion using a bowl full of marbles
and a ball of twine.  I've had a confederate ``interrupt'' class with
an apparently inane request which later turned out to illustrate a key
concept.  I use slides when necessary but keep them as simple as
possible, with very little text and lots of illustrations.  In short,
I try to stimulate many different modes of learning, and help students
develop intuition for big ideas, rather than getting lost in a maze of
facts.

Second, I emphasize self-discovery. Students remember ideas they
discover for themselves, because they understand the motivation and
how the idea fits into a larger context. I didn't tell the middle
school students that they were learning about finite automata---I just
let them play! The sorts of content-driven, lecture-based courses I
teach can make this difficult, but I try my best to engage students'
faculties of discovery by assigning open-ended projects, encouraging
questions, and interspersing lectures with small-group or individual
problem solving sessions.  I often introduce new topics in the form of
a challenging problem to be solved, then sit back and watch the
students hash out a solution among themselves.  Ten minutes of false
starts, arguments, insights, and eventual triumph are worth more than
ten hours of lecturing!

In an environment emphasizing creativity, exploration, and
self-discovery, failure is an unavoidable---even
necessary---concomitant.  I therefore gravitate towards
non-traditional methods of assessment that encourage and allow for
both failure and unhurried reflection, such as take-home, open-book
tests; interactive, oral examinations; and group projects.  I also
recognize my own constant failings as a teacher, as I explore subjects
along with my students and discover (sometimes successfully and
sometimes less so) how best to guide them.  This recognition informs
how I strive to interact with students experiencing failure of one
sort or another, from a missed question to a missed semester: fairly,
not shying away from the failure or its consequences, but with
compassion, grace, and second chances.
%The students who succeed are the ones who take on something
%unfamiliar to them, explore it

Above all, I try to convey an appreciation for beautiful ideas,
not to receptacles of knowledge but to human beings. A student will
almost certainly forget all the details of a proof or definition or
technique.  But if they remember the beautiful nugget of insight it
represents, and where it fits within the larger context of the
subject---then I will have truly taught them something.

% In college I took a number theory course which strongly influenced my
% ideas about teaching.  
% 
% Although I have never done so, I would also love to try
% teaching a course using a modification of the Moore method, pioneered
% by Robert Lee Moore at the beginning of the 20th century while a
% mathematics professor at the University of Pennsylvania. He explicitly
% taught his students as little as possible, only stating definitions
% and theorems and having the students work out the proofs for
% themselves. I once took an undergraduate number theory course taught
% in a similar style, and I still remember more from that course than
% from four or five other average courses put together!


\end{document}
