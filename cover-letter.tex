\documentclass{letter}

\usepackage{url}

\newif\ifteaching
\newif\ifresearch

%%% Parameters %%%%%%%%%%%%%%%%%%%%%%%%%%%%%%%%%%%
\teachingtrue
\researchtrue
\def\addressee{Whomever}
\def\thereaddress{Search Committee \\ 123 Search Committee Lane \\ El
  Frijoles, TX 12345}
\def\position{the position of Head Frobnosticator}
\def\foundon{I found on \url{monster.com}}
\def\opportunity{I am excited by the opportunity to bring additional
  prestige to your institution.}
%%%%%%%%%%%%%%%%%%%%%%%%%%%%%%%%%%%%%%%%%%%%%%%%%%

\signature{Brent A. Yorgey}
\address{257 S 46th St. Apt. 1 \\ Philadelphia, PA 19139}

\begin{document}

\begin{letter}{\thereaddress}

\opening{Dear \addressee,}

I am writing to apply for \position, which \foundon.  My strong
teaching record, [XXX describe research program in some
memorable/intriguing phrase?], and ZZZ make me an excellent
fit. \opportunity

\ifresearch
Broadly speaking, my research is about connecting theory and practice:
I love thinking deeply about beautiful mathematical abstractions and
then applying them to get real things done more powerfully, elegantly,
and efficiently.  I am also passionate about the centrality of
community in research---both in terms of collaboration and
communicating results via publishing and teaching.

Concretely, my research is in two main areas. The first is in the
intersection of combinatorics and programming languages.  My
dissertation focuses on the theory of \emph{combinatorial species}---a
unified framework for describing and analyzing a wide range of
combinatorial structures---and specifically on bridging the gap
between the theory's origins in pure combinatorics and its application
to the theory and practice of programming.  Combinatorial species have
the potential to revolutionize the way we reason about and work with
data structures, leading to increased expressivity and precision.
Species can unify ``labelled'' structures (such as arrays and finite
maps) with more traditional ``algebraic'' types (such as lists and
binary trees) under the same framework; at the same time, they provide
a framework for reasoning about issues of memory allocation and
layout, giving an ``end-to-end'' framework that allows reasoning about
everything from high-level properties of data structures down to
practical implementation issues.  There is plenty left to explore, and
I am excited to continue the work begun in my dissertation---for
example, investigating the use of species to work with
\emph{symmetric} data structures such as bags and cycles, or how the
closely related theory of \emph{generating functions} can be
interpreted computationally to encompass a wide range of algorithms
(serialization, deserialization, enumeration, random generation\dots)
over data structures.

My second main area of research is in the design and implementation of
domain-specific languages.  For the past five years I have been
developing a domain-specific language, embedded in Haskell, for
describing vector graphics
(\url{http://projects.haskell.org/diagrams/}).  My focus has been on
discovering fundamental abstractions through careful domain analysis,
resulting in an expressive, powerful language.  I have a long list of
potential projects in the same vein\ifteaching---many of which are
ripe for student involvement---\else, \fi including: developing good
abstractions and language constructs for animation and interactivity;
building user interfaces offering dual, editable, bidirectionally
connected views of code and graphical output; designing and
implementing ``natural'' DSLs for identifying subparts of a diagram
(\emph{e.g.} ``the third red triangle''); and building tools to
automatically create ``language levels'' for complex embedded DSLs by
selective suppression of polymorphism.  More generally, I am also
interested in applying techniques of domain analysis and DSL design to
other areas of computer science.  \fi % research

\ifteaching
I have extensive teaching experience, including two years as a high
school teacher and a number of courses at Penn as both a teaching
assistant and primary instructor.  Most notably, I designed and taught
a full-credit undergraduate course titled ``The Art of Recursion''
XXX.  Popular.  YYY need to say something place-specific about
teaching, perhaps.
\fi % teaching

% XXX something place-specific here?

I am nearing completion of my PhD and expect to finish in August 2014.
XXX --- talk about progress?  I can be reached at
\textsf{byorgey@gmail.com} or \textsf{215 350 4532}.  I am happy to
send any additional materials the committee might require such as
teaching evaluations.

Thank you for your consideration.  I look forward to hearing from you.

\closing{Sincerely,}

\end{letter}

\end{document}
