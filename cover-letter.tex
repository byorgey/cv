% \documentclass{letter}

% \usepackage{url}

% \newif\ifteaching
% \newif\ifresearch

%%% Parameters %%%%%%%%%%%%%%%%%%%%%%%%%%%%%%%%%%%
% \teachingfalse
% \researchtrue
% \def\salutation{Dear Sir or Madam}
% \def\thereaddress{Search Committee \\ 123 Search Committee Lane \\ El
%   Frijoles, TX 12345}
% \def\position{the position of Head Frobnosticator}
% \def\foundon{I learned about through a posting on the CRA website}
% \def\opportunity{I am excited by the opportunity to\dots}
% \def\placespecific{On a personal note\dots}
%%%%%%%%%%%%%%%%%%%%%%%%%%%%%%%%%%%%%%%%%%%%%%%%%%

\widowpenalty 10000

\signature{Brent A. Yorgey}
\address{257 S 46th St. Apt. 1 \\ Philadelphia, PA 19139}

\begin{document}

\begin{letter}{\thereaddress}

\opening{\salutation:}

I am writing to apply for \position, which \foundon.  \opportunity

\ifresearch
Broadly speaking, my research is about connecting theory and practice:
I love thinking deeply about beautiful mathematical abstractions and
then applying them to get real things done more powerfully, elegantly,
and efficiently.  Specifically, my research is in two main areas: the
first is in the intersection of combinatorics and programming
languages, turning abstract combinatorial results into practical
programming tools; the second is the design and implementation of
domain-specific languages (DSLs). I am also passionate about the
centrality of community in research---both in terms of collaboration
and communicating results via publishing and teaching.

My dissertation focuses on the theory of \emph{combinatorial
  species}---a unified framework for describing and analyzing a wide
range of combinatorial structures---and specifically on bridging the
gap between the theory's origins in pure combinatorics and its
application to the theory and practice of programming.  Combinatorial
species have the potential to revolutionize the way we reason about
and work with data structures, leading to increased expressivity and
precision.  Species can unify ``labelled'' structures (such as arrays
and finite maps) with more traditional ``algebraic'' types (such as
lists and binary trees) under the same framework; at the same time,
they provide the ability to reason about memory allocation and layout,
giving an ``end-to-end'' framework that allows reasoning about
everything from high-level properties of data structures to practical
implementation issues.  There is plenty left to explore, and I am
excited to continue the work begun in my dissertation---for example,
investigating the use of species to work with \emph{symmetric} data
structures such as bags and cycles, or how the closely related theory
of \emph{generating functions} can be interpreted computationally to
encompass a wide range of algorithms (serialization, deserialization,
enumeration, random generation\dots) over data structures.

For the past five years I have also been developing a domain-specific
language, embedded in Haskell, for describing vector graphics
(\url{http://projects.haskell.org/diagrams/}).  My focus has been on
discovering fundamental abstractions through careful domain analysis,
resulting in an expressive, powerful language. For example, diagrams
are represented using a novel tree structure based on the theory of
monoids and monoid actions, enabling many high-level operations on
diagrams such as juxtaposition, finding boundary points, and inclusion
testing.  I have ideas for many more projects in a similar
vein\ifteaching---many of which are ripe for undergraduate
involvement---\else, \fi including: developing good abstractions and
language constructs for animation and interactivity; building user
interfaces offering dual, editable, bidirectionally connected views of
code and graphical output; designing and implementing ``natural'' DSLs
for identifying subparts of a diagram (\emph{e.g.} ``the third red
triangle''); and building tools to automatically create ``language
levels'' for complex embedded DSLs by selective suppression of
polymorphism.  More generally, I am also interested in applying
techniques of domain analysis and DSL design to other areas of
computer science, in collaboration with researchers specializing in
those domains.  \fi % research

\ifteaching
I have extensive teaching experience, including two years as a high
school teacher and a number of courses at Penn as both a teaching
assistant and primary instructor.  Most notably, I designed and taught
a full-credit undergraduate course titled ``The Art of Recursion''
XXX.  Popular.  YYY need to say something place-specific about
teaching, perhaps.
\fi % teaching

\placespecific

I am nearing completion of my PhD and expect to finish in August 2014.
I have previously published an expository paper on the theory of
species for functional programmers, which will form the basis of a
chapter; I am currently preparing a paper for submission to MSFP 2014
with a substantial amount of technical development, which will form
the basis for another.  There is still work remaining to be done, but
I am making excellent progress and well on my way to completion. I
can be reached at \textsf{byorgey@gmail.com} or \textsf{215 350 4532}.
I am happy to send any additional materials the committee might
require\ifteaching, such as teaching evaluations\fi.

Thank you for your consideration.  I look forward to hearing from you.

\closing{Sincerely,}

\end{letter}

\end{document}
