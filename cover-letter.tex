% \documentclass{letter}

% \usepackage{url}
% \usepackage{etoolbox}
% \usepackage[noheadfoot,margin=1.5in]{geometry}

% \pagestyle{empty}

% \newtoggle{teaching}
% \newtoggle{research}

%%% Parameters %%%%%%%%%%%%%%%%%%%%%%%%%%%%%%%%%%%
% \toggletrue{teaching}
% \toggletrue{research}
% \def\salutation{Dear Sir or Madam}
% \def\thereaddress{Search Committee \\ 123 Search Committee Lane \\ El
%   Frijoles, TX 12345}
% \def\position{the position of Head Frobnosticator}
% \def\foundon{I learned about through a posting on the CRA website}
% \def\opportunity{I am excited by the opportunity to\dots}
% \def\placespecific{On a personal note\dots}
%%%%%%%%%%%%%%%%%%%%%%%%%%%%%%%%%%%%%%%%%%%%%%%%%%

\newcommand{\ronly}[1]{\ifboolexpr{togl{research}}{#1}{}}
\newcommand{\tonly}[1]{\ifboolexpr{togl{teaching}}{#1}{}}

\widowpenalty 10000

\signature{Brent A. Yorgey}
\address{Brent A. Yorgey \\ Department of Computer Science \\ Williams
  College \\ 47 Lab Campus Drive, TPL
  314 \\ Williamstown, MA 01267}

\begin{document}

\begin{letter}{\thereaddress}

\opening{\salutation:}

I am writing to apply for \position\foundon.  \opportunity

Broadly speaking, both my research and teaching are about connecting
theory and practice: I love developing insights into beautiful
mathematical abstractions, applying them to get real things done
powerfully, elegantly, and efficiently, and communicating them in
relevant, engaging ways.  \ronly{Specifically, my research is in two
  main areas: the first is in the intersection of combinatorics and
  programming languages, turning abstract combinatorial results into
  practical programming tools; the second is the design and
  implementation of domain-specific languages (DSLs). }I am
\ronly{also }passionate about the centrality of community to research
and teaching---collaboration with students and colleagues is one of my
main sources of joy as a computer scientist.

\ifboolexpr{ togl{research} }{

  My dissertation focuses on the theory of \emph{combinatorial
    species}---a unified framework for describing and analyzing a wide
  range of combinatorial structures---and specifically on bridging the
  gap between the theory's origins in pure combinatorics and its
  application to the theory and practice of programming.
  Combinatorial species have the potential to revolutionize the way we
  reason about and work with data structures, leading to increased
  expressivity and precision.  \ifboolexpr{togl{teaching}}{My research
    statement contains additional details about the work carried out
    in my dissertation and my plans for future work.}{Species can
    unify ``labelled'' structures (such as arrays and finite maps)
    with more traditional ``algebraic'' types (such as lists and
    binary trees) under the same framework; at the same time, they
    provide the ability to reason about memory allocation and layout,
    giving an ``end-to-end'' framework that allows reasoning about
    everything from high-level properties of data structures to
    practical implementation issues.  There is plenty left to explore,
    and I am excited to continue the work begun in my
    dissertation---for example, investigating the use of species to
    work with \emph{symmetric} data structures such as bags and
    cycles, or how the closely related theory of \emph{generating
      functions} can be interpreted computationally to encompass a
    wide range of algorithms (serialization, deserialization,
    enumeration, random generation\dots) over data structures.}

  For the past six years I have also been developing a
  domain-specific language, embedded in Haskell, for describing vector
  graphics (\url{http://projects.haskell.org/diagrams/}).  My focus
  has been on discovering fundamental abstractions through careful
  domain analysis, resulting in an expressive, powerful
  language. \ifboolexpr{togl{teaching}}{I have many ideas for future
    projects in this area---many of which are particularly conducive
    to undergraduate involvement---described in more detail in my
    research statement.}{For example, diagrams are represented using a
    novel tree structure based on the theory of monoids and monoid
    actions, enabling many high-level operations on diagrams such as
    juxtaposition, finding boundary points, and inclusion testing.  I
    have ideas for many more projects in a similar vein, including:
    developing good abstractions and language constructs for animation
    and interactivity; building user interfaces offering dual,
    editable, bidirectionally connected views of code and graphical
    output; designing and implementing ``natural'' DSLs for
    identifying subparts of a diagram (\emph{e.g.} ``the third red
    triangle''); and building tools to automatically create ``language
    levels'' for complex embedded DSLs by selective suppression of
    polymorphism.}  More generally, I am also interested in applying
  techniques of domain analysis and DSL design to other areas of
  computer science, in collaboration with researchers specializing in
  those domains.

} % research
{}

\tonly{

  Teaching is a way of life for me: it is a central and inextricable
  part of my research and creative process. I enjoy writing blog posts
  and papers, giving talks, and mentoring, in addition to designing
  and teaching courses. I care deeply about my students as whole
  persons, and work to make my courses into spaces where they feel
  welcomed and challenged.  I am also constantly in search of new and
  creative ways to communicate ideas effectively.  I particularly love
  taking abstract, beautiful mathematical ideas and finding intuitive
  ways to explain them, helping students make connections between
  topics.  In the process, I usually end up with better understanding
  and intuition myself, and am inspired to explore new and different
  questions in my research.

  As an example of some of the above themes, at the University of
  Pennsylvania I designed and taught a full-credit undergraduate
  course entitled ``The Art of Recursion'', which explored induction,
  the lambda calculus, equivalence of recursion and iteration, tail
  call elimination, lazy infinite data structures, and fixpoint
  theory, among other topics.  Most of these were not otherwise
  covered in Penn's undergraduate curriculum, so I saw this as an
  opportunity to make a significant contribution to undergraduate
  education at Penn.  Besides being beautiful and mind-expanding, such
  topics are increasingly relevant given industry trends towards
  functional programming, as reflected in the inclusion of functional
  programming as a core topic in the new ACM curriculum guidelines.  I
  structured the course in an unusual way---alternating lectures with
  student presentations---in an effort to foster collaboration and
  ownership of the course material among the students; more detail can
  be found in my teaching statement.  The course received high ratings
  and rave reviews from students, who described it in terms like
  ``best course at Penn so far'', ``never have I seen a class so
  engaged'', ``this class taught me new ways to think'', and praised
  my teaching as ``stimulating'' and ``hands-down the best instructor
  I've had in my four years at Penn''.

} % tonly

\placespecific

I recently defended my dissertation, completed under the supervision
of Dr. Stephanie Weirich, and expect to officially receive my PhD in
December 2014.  I can be reached by email at
\textsf{byorgey@gmail.com} or by phone at \textsf{215 350 4532}.

Thank you for your consideration.  I look forward to hearing from you.

\closing{Sincerely,}

\end{letter}

\end{document}
