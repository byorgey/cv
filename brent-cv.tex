% -*- compile-command: "rubber -d brent-cv.tex" -*-

\documentclass[12pt]{article}

\usepackage{brent}
\usepackage{hyperref}
% \usepackage{url}
\usepackage{titlesec}
\usepackage{hanging}
\usepackage{fullpage}
% \usepackage[noheadfoot]{geometry}

\pagestyle{empty}

\titleformat{\section}{}{}{}{\bfseries\uppercase}
\titlespacing*{\section}{0pt}{*2}{0pt}

\setlength{\parskip}{0pt}

\newcommand{\cvitem}{\par\hangpara{2em}{1}}

\begin{document}

\begin{center}
  {\huge Brent A. Yorgey} \\
  \emph{Curriculum Vit\ae} \\
  \today
\end{center}

\noindent
\parbox{3.5in}{
Hendrix College \\
Dept.\ of Mathematics \& Computer Science \\
1600 Washington Ave \\
Conway, AR 72032
}
\hfill
\parbox{3in}{
\texttt{byorgey@gmail.com} \\
\texttt{yorgey@hendrix.edu} \\
\url{http://ozark.hendrix.edu/~yorgey}
}
\medskip

\hrule

% XXX invited observer, IFIP WG 2.1 ??

\section*{Research interests}
Programming languages, functional programming, (embedded)
domain-specific languages, combinatorics, category theory, type
theory, dependent type systems

\section*{Education}
\cvitem
Ph.D. in Computer Science, University of
Pennsylvania, 2008--2014 \\ Dissertation: \emph{Combinatorial
  Species and Labelled Structures} \\ Advisor: Dr.\ Stephanie Weirich

\cvitem B.A. in Computer Science, \latin{summa cum laude}, Williams
College, June 2004

\section*{Awards and Honors}
\cvitem Harvey Fellow
\cvitem Penn Prize for Excellence in Teaching by Graduate Students, 2012
\cvitem Teaching Practicum Award, University of Pennsylvania, 2010
\cvitem Sam Goldberg Colloquium Prize in Computer Science, Williams
College, 2004
\cvitem Phi Beta Kappa
\cvitem Milken Scholar
\cvitem National Merit Scholar

\section*{Professional experience}
\cvitem
Associate Professor, August 2021--
\cvitem
Assistant Professor, July 2015--July 2021\\
Department of Mathematics and Computer Science \\
Hendrix College, Conway, AR

\cvitem
Visiting Assistant Professor \\
Department of Computer Science \\
Williams College, Williamstown, MA, July 2014--June 2015

\cvitem
Research/Teaching Assistant \\
University of Pennsylvania, Philadelphia, PA, August 2008--June 2014

\cvitem
Research Intern \\
Microsoft Research, Cambridge, England, June--August 2010 \\
Worked on an experimental extension to the Glasgow Haskell Compiler with
Simon Peyton Jones and Dimitrios Vytiniotis.

\cvitem
Software Developer \\
Ascella Technologies / CGI Federal, Washington,
DC, July 2006--July 2008

\cvitem
Private Math Tutor \\
Washington, DC, September 2006--June 2008

\cvitem
Computer Science and Mathematics Teacher \\
Woodrow Wilson Senior High School, Washington, DC, August 2004--June 2006

\cvitem
Research Assistant \\
University of Maryland, College Park, MD, Spring 2000 \\
Created real-time data compression utilities for experiments in
molecular physics using high-speed cameras.

\section*{Academic service}

\cvitem Coach, Hendrix competitive programming team, 2015--present

\cvitem CCSC-MidSouth Student Programming Contest co-chair, '18, '19, '20, '21

\cvitem Co-Chair, ICFP Artifact Evaluation Committee '20, '21

\cvitem Program committees: Haskell Implementors' Workshop '19,
CCSC-MS '19, Haskell '18, CCSC-MS '18, TyDe '17 (co-chair), TyDe '16,
CLA '16, TFPIE '16, FARM '14, Haskell '12

\cvitem General chair, Workshop on Functional Art, Music, Modelling
and Design (FARM), September 2018

\cvitem Steering Committee, Workshop on Type-Driven Development
(TyDe), 2017--2019

\cvitem Steering Committee, Workshop on Functional Art, Music,
Modeling, and Design (FARM), November 2014--

\cvitem Publicity chair, 2013 Workshop on Functional Art, Music,
  Modeling, and Design (FARM '13) \\
  Co-organizer, with Paul Hudak and Conal Elliott, of a new workshop
  bringing together academics and practitioners interested in
  applications of functional programming in art and design.

\cvitem Haskell core libraries committee, June 2013--June 2014

\cvitem Coordinator, Penn PLClub, June 2012--July 2014

\cvitem Editor, \emph{The Monad.Reader}, October 2009--October 2011 \\
A free electronic magazine about functional programming, targeted at
the Haskell community.

\cvitem Organizer, Hac $\varphi$ (July 2009, May 2010, July 2011,
August 2012, June 2013) \\
Open three-day meetings for collaboration on projects using the functional
programming language Haskell, with around 30 attendees.

\cvitem Editor, \emph{Haskell Weekly News}, June 2008--August 2009 \\
Collected and published a weekly gathering of news items from the
Haskell programming language community.

\section*{Community service}

\cvitem Elder, Christ Church Conway, January 2020--present

\cvitem Board member, Arkansas Asset Builders, December
2016--August 2018 \\
Volunteered on the board of a local nonprofit organization providing
free volunteer-prepared tax returns and financial literacy training to
members of the community.

\cvitem Volunteer tax preparer, Arkansas Asset Builders,
February--April 2017

\cvitem Haskell.org committee, October 2012--October 2014 \\
Helped set policy and oversee use of funds for Haskell open-source
community infrastructure.

\cvitem Penn Alexander middle school math club, October 2009--November
2010 \\
Volunteered to help lead middle school students in a variety of fun
and engaging mathematical explorations.

\section*{Refereed Publications}

\cvitem Brent A. Yorgey and Kenneth Foner. What's the difference? A
functional pearl on subtracting bijections.  In \emph{Proceedings of
  the ACM on Programming Languages} vol. 2 issue ICFP, September 2018,
  p. 101.

\cvitem Satvik Chauhan, Piyush P. Kurur, and Brent A. Yorgey. How to
twist pointers without breaking them. In \emph{Proceedings of the 9th
  International Symposium on Haskell} (Haskell '16), pp. 51--61.

\cvitem Ryan Yates and Brent A. Yorgey. Diagrams: A Functional EDSL
for Vector Graphics. In \emph{Proceedings of the 3rd Workshop on
  Functional Art, Music, Modeling and Design} (FARM '15), pp. 2--3.
  Demo/tutorial abstract.

\cvitem Dan Piponi and Brent A. Yorgey.  Polynomial Functors
Constrained by Regular Expressions.  In \emph{Proceedings of the 12th
  International Conference on the Mathematics of Program Construction}
(MPC '15), pp. 113--136.

\cvitem Brent A. Yorgey. Monoids: Theme and Variations
(\emph{Functional Pearl}).  In \emph{Proceedings of the 5th ACM
  SIGPLAN Symposium on Haskell} (Haskell '12, acceptance rate 41\%),
pp. 105--116.

\cvitem Brent A. Yorgey, Stephanie Weirich, Julien Cretin, Simon
Peyton Jones, Dimitrios Vytiniotis, and Jos\'e Pedro
Magalh\~aes. Giving Haskell a Promotion. In \emph{Proceedings of the
  8th ACM SIGPLAN Workshop on Types in Language Design and
  Implementation} (TLDI '12), pp. 53--66.

\cvitem Stephanie Weirich, Brent A. Yorgey, and Tim Sheard. Binders
Unbound. In \emph{Proceedings of the 16th ACM SIGPLAN International
  Conference on Functional Programming} (ICFP '11, acceptance rate
  36\%), pp. 333--345.

\cvitem Brent A. Yorgey. Species and Functors and Types, Oh
  My! (\emph{Functional Pearl}). In \emph{Proceedings of the 3rd ACM SIGPLAN
  Symposium on Haskell} (Haskell '10, acceptance rate 39\%), pp. 147--158.

% \section*{In preparation}

% \cvitem Brent A. Yorgey, Jacques Carette, and Stephanie Weirich.
% Combinatorial Species and Labelled Structures (working title). \\
% Using the theory of combinatorial species as a foundational
%   basis for a richer, unified notion of data types in programming
%   languages.

\section*{Books}
\cvitem Benjamin C. Pierce, Chris Casinghino, Marco Gaboardi, Michael Greenberg,
C{\v a}t{\v a}lin Hri{\c t}cu, Vilhelm Sj\"oberg, and Brent Yorgey. \emph{Software
  Foundations}. \url{http://www.cis.upenn.edu/~bcpierce/sf/}.

\section*{Other Publications}
\cvitem Brent Yorgey. \texttt{blog :: Brent ->
  [String]}. \url{http://byorgey.wordpress.com}.  \\
A blog aimed at the academic community, for discussing current ideas
and research.  June 2007--present.

\cvitem Brent Yorgey. \emph{The Math Less
  Traveled}. \url{http://www.mathlesstraveled.com}. \\
A blog aimed at a broad audience, especially high school students,
exploring beatiful ideas in mathematics.  March 2006--present.

\cvitem Brent Yorgey. \emph{Catsters
  guide}. \url{https://byorgey.wordpress.com/catsters-guide-2/}. \\
An online guide to the series of 70-odd category theory video lectures
put out by ``The Catsters'' (Eugenia Cheng and Simon Willerton).

\cvitem Brent Yorgey. \emph{The Typeclassopedia}. In: The
Monad.Reader, Issue 13, March 2009.

\cvitem Brent Yorgey. \emph{Generating Multiset Partitions}.  In: The
Monad.Reader, Issue 8, September 2007.


\section*{Talks}

\cvitem \emph{Disco: A Functional Teaching Language for Discrete
  Mathematics}. \\ IFIP Working Group 2.1, Meeting \#78, Taiwan. March 2019.
\cvitem \emph{What's the Difference? A Functional Pearl on Subtracting
    Bijections}. \\ International Conference on Functional Programming
    (ICFP). September 26, 2018.
\cvitem \emph{Graph Coloring with a SAT Solver}. \\
    Consortium for Computing Sciences in Colleges Mid-South.  April 6, 2018.
\cvitem \emph{Explaining Type Errors}. \\
    With Richard Eisenberg and Harley Eades.  Off The Beaten Track
    (OBT). January 13, 2018.
\cvitem \emph{Diagrams: A Functional EDSL for Vector Graphics}. \\
    With Ryan Yates. 3rd Workshop on Functional
    Art, Music, Modeling and Design (FARM).  September 5, 2015.
\cvitem \emph{Polynomial Functors Constrained by Regular
    Expressions}. \\ Invited talk at University of Kansas. July 16, 2015.
    \\ Mathematics of Program Construction (MPC),
    K\"onigswinter, Germany. June 29, 2015.
    \\ Faculty math seminar at Williams College. March 13, 2015.
\cvitem \emph{Derivatives of Data Types, via Regular
    Expressions}. \\ Invited talk at Wesleyan University. May 5, 2015.
\cvitem \emph{Building Domain-Specific Languages and Tools}. \\
    Hendrix College. February 27, 2015. \\
    Grinnell College. February 9, 2015.
\cvitem \emph{\texttt{Diagrams}: Declarative Vector Graphics in
    Haskell}. \\ Invited talk at New York Haskell Users' Group. November
    20, 2013.
\cvitem \emph{Trees and Things (with Semirings!)}. \\ Invited talk at Houghton
    College. October 29, 2013. \\
    Williams College. February 26, 2014.
\cvitem \emph{Functional Active Animation}. \\ Workshop on Functional
    Art, Music, Modeling and Design (FARM).  September 28, 2013.
\cvitem \emph{Monoids: Theme and Variations}. \\ Haskell Symposium.
    September 13, 2012.
\cvitem \emph{Embedded, functional, compositional drawing}. \\ Invited
    talk at Williams College. April 13, 2012.
\cvitem \emph{Giving Haskell a Promotion}. \\ Workshop on Types in
    Language Design and Implementation (TLDI). January 28, 2012.
\cvitem \emph{Typed type-level programming with GHC}. \\ Haskell
    Implementors' Workshop. October 1, 2010.
\cvitem \emph{Species and Functors and Types, Oh My!} \\ Haskell
    Symposium. September 30, 2010.
\cvitem \emph{Random testing---and beyond! with combinatorial
    species}. \\ Invited talk at University of Kansas.  November 24, 2009.
\cvitem \emph{Executable Mathematics: a Whirlwind Introduction to
    Haskell}. \\ Williams College.  June 8, 2008.
\cvitem \emph{xmonad: a Haskell Success Story}. \\ FringeDC, Washington,
    DC. March 22, 2008.

\section*{Teaching}

\cvitem
Hendrix College, 2015-- \\
CSCI 150, Foundations of Computer Science (F'15, S'16, S'17, S'18,
F'18, S'19, F'19, S'21) \\
CSCI 151, Data Structures (F'16, F'17, S'19) \\
CSCI 382, Algorithms (S'16, S'17, F'17, F'18, F'19, F'20) \\
CSCI 360, Programming Languages (F'16, F'18, S'21) \\
CSCI 365, Functional Programming (S'16, S'18, S'20) \\
CSCI 410, Senior Seminar (F'16, F'17, F'19, F'20) \\
MATH 240, Discrete Mathematics (S'20, S'21) \\
LBST 150J, The Engaged Citizen: The Art and Science of Creativity
(F'20) \\
LBST 101, Explorations (F'18, F'20)

\cvitem
Williams College, 2014--2015 \\
CS 134, Introduction to Computer Science (co-taught with Bill Lenhart,
F'14) \\
CS 136, Data Structures and Advanced Programming (S'15) \\
CS 354, Functional Programming and the Art of Recursion (F'14)

\cvitem
University of Pennsylvania, 2008--2014 \\
Earned teaching certificate through Penn Center for Teaching and
Learning \\
CIS 194, Introduction to Haskell (Course designer and instructor ---
F'10, S'12, S'13) \\
CIS 399, The Art of Recursion (Course designer and instructor ---
F'12) \\
CIS 500, Software Foundations (TA --- S'10, S'11) \\
CIS 120, Programming Languages and Techniques I (TA --- F'09)

\cvitem
Correspondence course in precalculus with two homeschool students
(Course designer and instructor --- 2008-2009)

\cvitem
Woodrow Wilson Senior High School, Washington, DC, 2004--2006 \\
Introduction to Computer Science ('04--'05, '05--'06) \\
AP Computer Science AB ('04--'05, '05--'06) \\
Honors Precalculus ('05--'06)

\cvitem
Williams College, Williamstown, MA, 2001--2004 \\
Discrete Mathematics (TA, S'01) \\
Computer Organization and Architecture (TA, F'01, F'02, F'03) \\
Abstract Algebra (TA, S'03) \\
Computational Geometry (TA, S'04) \\

% \section*{Mentoring}

% XXX John McAvey, My Nguyen, Mitchell Sharp

% XXX Google Summer of Code mentoring

\section*{Personal}

Excellent classical pianist.  Enjoy playing bridge and go.  Good
reading knowledge of ancient Greek and biblical Hebrew.  Ranked 14th
on competitive programming website \url{http://open.kattis.com}.

\end{document}
