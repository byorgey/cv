\documentclass{article}

\pagestyle{empty}

\begin{document}

\noindent Brent Yorgey \\
Research statement \\
\today
\bigskip

My primary research interests include functional programming
languages, embedded domain-specific languages, type systems, and
combinatorics, and in ways that these topics can contribute to other
areas of computer science.  More broadly, running through all of my
research are three major themes which shape and direct my efforts:

\begin{itemize}
\item I love the process of designing beautiful, coherent visions and
  then following through on the details to make those visions into
  reality, which I call \textbf{architecting}.  I am not interested in
  theory just for theory's sake, nor in just hacking out details, but
  in a harmonious synthesis of the two.
\item I am highly motivated by \textbf{beauty} in its many forms.  I
  learn new things because they are beautiful; I strive for beauty in
  all my creative output; I am motivated to communicate an
  appreciation for beauty in my teaching.  Beauty also correlates
  with generality: the most beautiful solutions are the ones that have
  cross-discipline applicability.
\item Architecting and appreciating beauty always happen in the
  context of community.  I love \textbf{communicating} ideas, through
  both teaching and writing, and see communication and collaboration
  as a central and necessary aspect of my research. XXX say something
  about advising undergrad projects?
 % In addition, many of my projects
 %  are intended to enable the communication of beautiful ideas in
 %  beautiful ways.
\end{itemize}

In what follows, I describe several current areas of research,
highlighting the ways they fit into the above themes, and elaborate on
planned future directions for my research.

\section*{Combinatorics}
\label{sec:combinatorics}

I am keenly interested in the intersection of combinatorics and
computation.  In particular, taking well-understood results in pure
combinatorics and ``porting'' them to a computational setting yields
both practical programming tools and new insight into the underlying
mathematics.

My dissertation research focuses on the \emph{theory of species},
which has been developed over the past three decades as a unifying
framework for enumerative combinatorics: discrete structures can be
described algebraically and systematically analyzed using generating
functions.  As a purely combinatorial theory, species are relatively
well-explored; but XXX.  The core of my dissertation is to port the
theory of species from set theory to constructive type theory, and to
then use them as a foundational basis for data structures in
programming languages. This results in a theory of \emph{labelled
  structures} which unifies algebraic data types (lists, binary trees,
\dots) and what one would more typically think of as labelled
structures (arrays, finite maps, \dots), leading to richer
expressivity and a conceptual framework for understanding existing
algorithms and data structures in new ways.

\subsection*{Future directions}

 The
theory of species can also serve as a foundation for understanding
structures with \emph{symmetry} (cycles, bags, \dots)

Generating functions. XXX This is a rich seam of material with plenty
of combinatorics left to ``mine'' for computational significance.

\section*{Embedded domain-specific languages}
\label{sec:edsls}

Almost any domain can benefit greatly from the principled design of
domain-specific languages and type systems.  Such languages enable XXX.

My current work on embedded domain-specific languages focuses on the
areas of graphics and animation.

\subsection*{Future directions}

Bidirectional UI.  More broadly applicable.  Note possibility of
student involvement.


\end{document}