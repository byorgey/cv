\documentclass{article}

\pagestyle{empty}

\begin{document}

\noindent Brent Yorgey \\
Research interests \\
\today
\bigskip

My primary research interests include functional programming
languages, combinatorics, embedded domain-specific languages, and type
systems.  More broadly, running through all of my research are three
major themes which shape and direct my efforts:

\begin{itemize}
\item I love the process
  of designing beautiful, coherent visions and then following through on
  the details to make those visions into reality, which I call \texttt{architecting}.  XXX say more
\item I am highly motivated by \texttt{beauty} in its many forms.  I
  learn new things because they are beautiful; I strive for beauty in
  all my creative output; I am motivated to communicate an
  appreciation for beauty in my teaching.
\item Architecting and appreciating beauty always happen in the
  context of a community.  I love \texttt{communicating} ideas,
  through both teaching and writing, and see communication as a
  central and necessary aspect of my research. XXX say something about
  advising undergrad projects?  In addition, many of my projects are
  intended to enable the communication of beautiful ideas in beautiful
  ways.
\end{itemize}

In what follows, I describe several current areas of research,
highlighting the above themes, and elaborate on future directions of
research.

\section*{Combinatorics}
\label{sec:combinatorics}

I am keenly interested in the intersection of combinatorics and
computation.  In particular, taking well-understood results in pure
combinatorics and ``porting'' them to a computational setting yields
both practical programming tools and new insight into the underlying
mathematics.

My dissertation research focuses on the \emph{theory of species},
which XXX as a unifying framework for enumerative combinatorics.  As a
purely combinatorial theory, species are relatively well-explored; but
XXX.  The core of my dissertation is to port the theory of species
from set theory to constructive type theory, and use species as a
basis for data structures in programming languages, resulting in a
theory of \emph{labelled structures} which unifies algebraic data
types (lists, binary trees, \dots) and what one would more typically
think of as labelled structures (arrays, finite maps, \dots).  The
theory of species can also serve as a foundation for understanding
structures with \emph{symmetry} (cycles, bags, \dots)

Generating functions. XXX This is a rich seam of material with plenty
of combinatorics left to ``mine'' for computational significance.

\section{Embedded domain-specific languages}
\label{sec:edsls}

\end{document}