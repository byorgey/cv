% -*- compile-command: "pdflatex research-statement.tex" -*-
\documentclass[12pt]{article}

\usepackage{fullpage}

\pagestyle{empty}

\begin{document}

\noindent Brent Yorgey \\
Research statement \\
\today
\bigskip

My primary research interests include functional programming
languages, embedded domain-specific languages, type systems, and
combinatorics, and ways that these topics can contribute to other
areas of computer science.  More broadly, running through all of my
research are three major themes which shape and direct my efforts:

\begin{itemize}
\item I am highly motivated by \textbf{beauty} in its many forms.  I
  learn new things because they are beautiful; I strive for beauty in
  all my creative output; I am motivated to communicate an
  appreciation for beauty in my teaching.  Beauty also correlates
  with generality: the most beautiful solutions are the ones that have
  cross-discipline applicability.
\item I love the process of designing beautiful, coherent visions and
  then following through on the details to make those visions into
  reality, which I call \textbf{architecting}.  I am not interested in
  theory just for theory's sake, nor in just hacking out details, but
  in a harmonious synthesis of the two.
\item Architecting and appreciating beauty always happen in the
  context of community.  I love \textbf{communicating} ideas, through
  both teaching and writing, and see communication and collaboration
  as a central and necessary aspect of my research.
  % XXX say something
  % about advising undergrad projects?
 % In addition, many of my projects
 %  are intended to enable the communication of beautiful ideas in
 %  beautiful ways.
\end{itemize}

In what follows, I describe several current areas of research,
highlighting the ways they fit into the above themes, and elaborate on
planned future directions for my research.

%% XXX actually highlight how they fit into the above themes!  And
%% highlight continuity between them: applying beautiful math to give
%% insight/solve problems/etc.

\section*{Combinatorics, computation, and data structures}
\label{sec:combinatorics}

I am keenly interested in the intersection of combinatorics and
computation.  In particular, taking results in pure combinatorics and
``porting'' them to a computational setting yields both practical
programming tools and new insight into the underlying mathematics.

My dissertation research focuses on the \textbf{theory of
  combinatorial species}, developed over the past three decades as a
unifying framework for enumerative combinatorics. As a purely
combinatorial theory, species are relatively well-explored, but there
are striking connections to the theory of algebraic data types which
remain largely unexplored and unexploited.  The core of my
dissertation is to port the theory of species from set theory to
constructive type theory, and to then use it as a foundational basis
for data structures in programming languages. One of the theory's
great strength lies in a coherent, precise description of the
relationship between \emph{labelled} and \emph{unlabelled} structures,
so this results in a theory of \textbf{labelled data structures} which
unifies algebraic data types (such as lists or binary trees) and what
one would more typically think of as labelled structures (vectors,
arrays, finite maps, \dots), leading to richer expressivity and a
conceptual framework for understanding existing algorithms and data
structures in new ways.  For example, it seems that the theory gives
an abstract yet precise way to think about issues of memory layout and
allocation, in, say, matrix algorithms.

\subsection*{Future directions}

Another strength of the theory of species is its ability to describe
structures with nontrivial symmetries, such as cycles or bags.
Computationally, these correspond to data structures where one
``abstract'' value can have multiple concrete representations which
the programmer should not be able to observe.  Such abstract data
types occur frequently, but are typically implemented in an ad-hoc
way, with no help from the compiler in ensuring that the
implementation is correct or that the abstract interface does not leak
representation detail.  Species may afford a framework for reasoning
about such abstract data structures, though the details are far from
obvious.  This is an area I have explicitly chosen not to explore in
my doctoral research, but which I am excited to ponder in the future.

Yet another key feature of the theory of species is its strong
connection to the theory of generating functions, which can be used to
summarize and extract information of interest about species.  This is
a rich seam of material with plenty of combinatorics left to ``mine''
for computational significance.  In particular, my hope is that many
algorithms of interest over data structures (enumeration, random
generation, serialization, deserialization\dots) can be reformulated
using the framework of generating functions, leading to new insights
and new algorithms.

Unfortunately, there is not much scope for undergraduate involvement
in these threads of my research, since attacking the interesting
questions typically requires background in combinatorics and/or
category theory that most undergraduates would not have.  In contrast,
however, there are many potential ways for undergraduates to become
involved in the other main area of my research, outlined in the next
section.

\section*{Domain-specific languages}
\label{sec:edsls}

Almost any domain can benefit greatly from the principled design of
\textbf{domain-specific languages and type systems}.  Such languages
enable more economical communication of problems and solutions in the
domain (with both computers and humans), and abstracting away
domain-irrelevant detail enables higher-level thinking and new
insights.  Moreover, the process of constructing domain-specific
languages itself often sheds new light on the domain under
consideration, since it exposes fundamental questions about the
meaning of entities and operations in the domain.

My current work on embedded domain-specific languages focuses on the
areas of graphics and animation.  For the past five years I have been
developing a domain-specific language, \texttt{diagrams}, for
describing \textbf{vector graphics}.  It is embedded in the Haskell
programming language, with emphases on expressivity, elegant design,
and careful analysis of the underlying domain, leading to the
application of mathematical abstractions capturing its essence.

\subsection*{Future directions}

I am currently collaborating on a domain-specific language for
constructing \emph{animations}, 

Interactivity.

Bidirectional UI.  More broadly applicable.  Note possibility of
student involvement.

Techniques for embedding.

Applying DSLs to other areas.

\end{document}