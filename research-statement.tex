% -*- compile-command: "pdflatex research-statement.tex" -*-
\documentclass[12pt]{article}

\usepackage{fullpage}
\usepackage{url}
\usepackage{natbib}

\pagestyle{empty}

\begin{document}

\noindent Brent Yorgey \\
Research statement \\
\today
\bigskip

My primary research interests include functional programming
languages, embedded domain-specific languages, type systems, and
combinatorics, and ways that these topics can contribute to other
areas of computer science.  More broadly, running through all of my
research are three major themes which shape and direct my efforts:

\begin{itemize}
\item I am highly motivated by \textbf{beauty} in its many forms.  I
  learn new things because they are beautiful; I strive for beauty in
  all my creative output; I am motivated to communicate an
  appreciation for beauty in my teaching.  Beauty also correlates
  with generality: the most beautiful solutions are the ones that have
  cross-discipline applicability.
\item I love the process of designing beautiful, coherent visions and
  then following through on the details to make those visions into
  reality, which I call \textbf{architecting}.  I am not interested in
  theory just for theory's sake, nor in just hacking out details, but
  in a harmonious synthesis of the two.
\item Architecting and appreciating beauty always happen in the
  context of community.  I love \textbf{communicating} ideas, through
  both teaching and writing, and see communication and collaboration
  as a central and necessary aspect of my research. Good research is
  not a solitary pursuit, but a dialogue among researchers---I am
  always on the lookout for other researchers and students to
  collaborate with.
\end{itemize}

In what follows, I describe several current areas of research,
highlighting the ways they fit into the above themes, and elaborate on
planned future directions for my research.

%% XXX actually highlight how they fit into the above themes!  And
%% highlight continuity between them: applying beautiful math to give
%% insight/solve problems/etc.

\section*{Combinatorics, computation, and data structures}
\label{sec:combinatorics}

I am keenly interested in the intersection of combinatorics and
computation.  In particular, taking results in pure combinatorics and
``porting'' them to a computational setting yields both practical
programming tools and new insight into the underlying mathematics.

My dissertation focuses on the \textbf{theory of
  combinatorial species}, developed over the past three decades as a
unifying framework for enumerative combinatorics. As a purely
combinatorial theory, species are relatively well-explored, but there
are striking connections to the theory of algebraic data types which
remain largely unexplored and unexploited. I have done some
expository work, both in a paper published in the Haskell
Symposium~\cite{yorgey2010species} and on my blog~\cite{blog}, making the
theory accessible to the average functional programmer.  The core of
my dissertation is to port the theory of species from set theory to
constructive type theory, and to then use it as a foundational basis
for data structures in programming languages. One of the theory's
great strengths lies in a coherent, precise description of the
relationship between \emph{labelled} and \emph{unlabelled} structures,
so this results in a theory of \textbf{labelled data structures} which
unifies algebraic data types (such as lists or binary trees) and what
one would more typically think of as labelled structures (vectors,
arrays, finite maps, \dots), leading to richer expressivity and a
conceptual framework for understanding existing algorithms and data
structures in new ways.  For example, the theory gives an abstract yet
precise way to think about issues of memory layout and allocation in
matrix algorithms.

\subsection*{Future directions}

Another strength of the theory of species is its ability to describe
\textbf{structures with nontrivial symmetries}, such as cycles or
bags.  Computationally, these correspond to data structures where one
``abstract'' value can have multiple concrete representations which
the programmer should not be able to distinguish.  Such abstract data
types occur frequently, but are typically implemented in an ad-hoc
way, with no help from the compiler in ensuring that the
implementation is correct or that the abstract interface does not leak
representation details.  Species may afford a framework for reasoning
about such abstract data structures, though the details are far from
obvious.

Yet another key feature of the theory of species is its strong
connection to the theory of \textbf{generating functions}, which can
be used to summarize and extract information of interest about
species.  This is a rich seam of material with plenty of combinatorics
left to ``mine'' for computational significance.  In particular, I
suspect that many algorithms of interest over data structures
(enumeration, random generation, serialization, deserialization\dots)
can be reformulated using the framework of generating functions,
leading to new insights and new algorithms.

\section*{Domain-specific languages and type systems}
\label{sec:edsls}

Almost any domain can benefit greatly from the principled design of
\textbf{domain-specific languages (DSLs) and type systems}.  Such
languages enable more economical communication of problems and
solutions in the domain, and abstracting away domain-irrelevant detail
enables higher-level thinking and new insights.  Moreover, the process
of constructing domain-specific languages itself often sheds new light
on the domain under consideration, since it exposes fundamental
questions about the meaning of entities and operations in the domain.

Type systems play an important role in the design of expressive DSLs.
Sme of my earlier work was concerned with extending Haskell's type
system to allow ``promotion'' of values to
types~\cite{Yorgey2012promotion}, a feature particularly useful for
DSLs embedded in Haskell.  That work has made a big impact in the
functional programming world, garnering nearly 50 citations over the
past two years.

My current work on embedded domain-specific languages focuses on the
areas of graphics and animation.  For the past six years I have been
developing a domain-specific language, \texttt{diagrams}, for
describing \textbf{vector graphics}
(\url{http://projects.haskell.org/diagrams}). It is embedded in the
Haskell programming language, with emphases on expressivity, elegant
design, and careful analysis of the underlying domain. This careful
analysis often leads to the discovery of mathematical abstractions
that elegantly capture the essence of some aspect of the domain.  For
example, as described in a Haskell symposium
paper~\cite{Yorgey2012monoids}, diagrams are represented using a novel
tree structure with both upwards- and downwards-accumulating
annotations, based on the theory of monoids and monoid actions.

\subsection*{Future directions}

I am especially excited about this area of research since it provides
\textbf{many potential avenues for student involvement}, across a wide
range of backgrounds and interests.  Students can get their feet wet
helping with the implementation of a DSL or brainstorming new
features, or more advanced students can explore type systems and the
mathematical semantics underlying new DSLs.

\texttt{diagrams} currently makes use of a preliminary domain-specific
language for constructing \textbf{animations} (and more generally, any
\emph{time-varying} values).  In collaboration with Andy Gill and Nick
Wu, I have been working on a complete redesign based on the theory of
2-categories---despite the rather abstract underpinnings, the
resulting API promises to be expressive, elegant, and practical.  The
insights we have gained are likely relevant for other time-centric
domains as well, such as music and signal processing.

Another direction in which I would like to extend the
\texttt{diagrams} framework is adding \textbf{interactivity}.
Currently, users can describe static or animated diagrams, but there
is no easy way to create diagrams which respond to input.  What are
the underlying semantics of iteractivity?  What would it look like to
have a convenient language in which to describe interactive things?
Part of the answer may lie in recent work on \emph{functional reactive
  programming}, but finding a natural way to integrate this work with
\texttt{diagrams} will likely raise new questions and lead to new
insights.

In addition to interactivity in the end result, what about
interactivity in the process of constructing diagrams itself?  The
idea is to develop a \textbf{bidirectional user interface} which
allows a user to construct a diagram either by writing code or by
interacting directly with the displayed diagram using their mouse,
either to edit the existing diagram or to draw new components.  In
either case, edits to one form (code or image) will be immediately
reflected in the other.  The new techniques needed to make such a
bidirectional user interface work should also be applicable in domains
other than drawing.

Finally, I am always on the lookout for other interesting domains
which could benefit from this kind of approach, and especially for
opportunities to collaborate with experts in those domains.

\bibliographystyle{plain}
\bibliography{research}

\end{document}